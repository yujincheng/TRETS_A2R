\documentclass[12pt]{paper}
\usepackage{geometry}
\usepackage{color}
\geometry{left=3.0cm, right=3.0cm, top=2.0cm, bottom=2.0cm}

\usepackage{booktabs} % For formal tables


% \usepackage[ruled]{algorithm2e} % For algorithms

\usepackage{cleveref}
\usepackage{epsfig}
\usepackage{float}
\usepackage{amsmath,amssymb,amsfonts}
\usepackage{algorithm}
\usepackage[noend]{algpseudocode}
\usepackage{bigstrut,multirow}
\usepackage{threeparttable}
\usepackage{subfigure}
\usepackage{url}

\newcommand{\rev}[1]{{{\color[rgb]{0,0,1}{#1}}}}
\newcommand{\reviewer}[1]{\section*{Reviewer #1}}
\newcommand{\answer}[1]{\noindent\textbf{Answer:} #1}
\newcommand{\comment}[1]{\noindent\textbf{#1}\\}


\title{Answer to Reviewers}

\begin{document}

\maketitle

We sincerely thank all the reviewers and editors for providing us so many valuable suggestions to improve the quality of this paper. All the review comments are addressed and answered. Please see the answers below and the corresponding contents in the paper.

\rev{The newly added contents and the major revised contents are marked in blue in the paper and in this letter.}

The major improvements are listed as follows:



\begin{itemize}
    \item We make the contributions of this work more clear and give a figure illustrating our workflow at the introduction part of this article. 

    \item We improve section 4 substantially. We add a more careful description with several figures to explain the cross-layer scheduling method.

    \item We add a section to describe our instruction set more clearly(Section 6), with details of each field. We also give code examples.
    
    \item We improve the evaluation section(Section 7). We discard some unnecessary results and add some new experiments to compare with previous work.

\end{itemize}

\reviewer{1}

This work deals with design and optimization of cross-layer CNN accelerator for FPGAs. Overall, it is a very good and complete work which seems, so far right on the State of the Art. General paper structure, description and results are fine. However, I have some comments I would really appreciate authors to address before publication (this is why I am issuing a 'minor review'). 

\comment{1.- First issue in the English language usage. I am not a native speaker but I've been able to find several mistakes, wrongly built sentences and improper uses of punctuation that make the paper difficult to read. I suggest a complete review of this, preferably by a native English speaker, or at least a very thorough and careful review of a very good English 'writer'. I have not few marks in my printed version, but I see no easy way to convey these to authors. However, as I say, a careful revision should suffice.}

\answer{Thanks for your advice. We have reviewed this paper according to all the comments you list below and make improvements. We have also gone through this paper to check spelling and grammatical errors. We also have invited a much more senior English writer to review and re-organize the language in this article.}

\comment{2.- There are many terms not properly defined. Although terms like CNN are widely known, other as RCNN, SSD, mAP (for the accuracy in Table 1), "constants" like $S$ in algorithm descriptions, NMS, etc....are not that known to the FPGA community. I kindly suggest authors to please define every term in the paper the first time it appears.}

\answer{I have added the full spelling of each abbreviations when it first appears in this paper, like (RCNN/SSD/NMS). At Section 2, I have added a paragraph to introduce the evaluation of an image detection algorithm. The evaluating indicator is Mean Average Precision(mAP). The $S$ in the code stands for stride. For easiler undertand 
}



\end{document}

